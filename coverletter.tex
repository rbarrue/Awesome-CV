%!TEX TS-program = xelatex
%!TEX encoding = UTF-8 Unicode
% Awesome CV LaTeX Template for Cover Letter
%
% This template has been downloaded from:
% https://github.com/posquit0/Awesome-CV
%
% Authors:
% Claud D. Park <posquit0.bj@gmail.com>
% Lars Richter <mail@ayeks.de>
%
% Template license:
% CC BY-SA 4.0 (https://creativecommons.org/licenses/by-sa/4.0/)
%


%-------------------------------------------------------------------------------
% CONFIGURATIONS
%-------------------------------------------------------------------------------
% A4 paper size by default, use 'letterpaper' for US letter
\documentclass[11pt, a4paper]{awesome-cv}

% Configure page margins with geometry
\geometry{left=1.4cm, top=.8cm, right=1.4cm, bottom=1.8cm, footskip=.5cm}

% Color for highlights
% Awesome Colors: awesome-emerald, awesome-skyblue, awesome-red, awesome-pink, awesome-orange
%                 awesome-nephritis, awesome-concrete, awesome-darknight
\colorlet{awesome}{awesome-skyblue}
% Uncomment if you would like to specify your own color
% \definecolor{awesome}{HTML}{CA63A8}

% Colors for text
% Uncomment if you would like to specify your own color
% \definecolor{darktext}{HTML}{414141}
% \definecolor{text}{HTML}{333333}
% \definecolor{graytext}{HTML}{5D5D5D}
% \definecolor{lighttext}{HTML}{999999}
% \definecolor{sectiondivider}{HTML}{5D5D5D}

% Set false if you don't want to highlight section with awesome color
\setbool{acvSectionColorHighlight}{true}

% If you would like to change the social information separator from a pipe (|) to something else
\renewcommand{\acvHeaderSocialSep}{\quad\textbar\quad}


%-------------------------------------------------------------------------------
%	PERSONAL INFORMATION
%	Comment any of the lines below if they are not required
%-------------------------------------------------------------------------------
% Available options: circle|rectangle,edge/noedge,left/right
% \photo[circle,noedge,left]{./examples/profile}
\name{Ricardo}{Barrué}
\position{PhD researcher at \href{https://www.lip.pt/?lang=en}{Laboratório de Instrumentação e Física Experimental de Partículas (LIP)} and \href{https://tecnico.ulisboa.pt/en/}{Instituto Superior Técnico (IST)}}
\address{LIP: Av. Prof. Gama Pinto, n.2, 1649-003 Lisboa, Portugal; IST: Av. Rovisco Pais, 1, 1049-001 Lisboa, Portugal}

\email{ricardo.barrue@cern.ch}
%\dateofbirth{January 1st, 1970}
% \homepage{www.posquit0.com}
\github{rbarrue}
\linkedin{ricardo-barrue}
\gitlab{rcoelhob}
\orcid{0000-0001-8985-5379}
% \gitlab{gitlab-id}
% \stackoverflow{SO-id}{SO-name}
% \twitter{@twit}
% \skype{skype-id}
% \reddit{reddit-id}
% \medium{madium-id}
% \kaggle{kaggle-id}
% \hackerrank{hackerrank-id}
% \googlescholar{googlescholar-id}{name-to-display}
%% \firstname and \lastname will be used
% \googlescholar{googlescholar-id}{}
% \extrainfo{extra information}

% \quote{``Be the change that you want to see in the world."}


%-------------------------------------------------------------------------------
%	LETTER INFORMATION
%	All of the below lines must be filled out
%-------------------------------------------------------------------------------
% The company being applied to
\recipient
  {Marisa Eisenberg, Tom Schwarz}
  {University of Michigan/MICOM \\ 500 S. State Street \\ Ann Arbor, MI 48109} 
% The date on the letter, default is the date of compilation
\letterdate{\today}
% The title of the letter
\lettertitle{Job Application for Postdoctoral Research Fellow}
% How the letter is opened
\letteropening{Dear Profs. Eisenberg and Schwarz,}
% How the letter is closed
\letterclosing{Sincerely,}
% Any enclosures with the letter
\letterenclosure[Attached]{Curriculum Vitae, Reference Statement}


%-------------------------------------------------------------------------------
\begin{document}

% Print the header with above personal information
% Give optional argument to change alignment(C: center, L: left, R: right)
\makecvheader[R]

% Print the footer with 3 arguments(<left>, <center>, <right>)
% Leave any of these blank if they are not needed
\makecvfooter
  {\today}
  {Ricardo Barrué ~~~·~~~ Cover Letter}
  {}

% Print the title with above letter information
\makelettertitle

%-------------------------------------------------------------------------------
%	LETTER CONTENT
%-------------------------------------------------------------------------------
\begin{cvletter}

I am writing to express my interest in applying for the postdoctoral position at the Michigan Public Health Integrated Center for Outbreak Analytics and Modeling (MICOM). I am finishing my PhD in experimental high-energy physics at the University of Lisbon (exp. finish data Feb. 2025), working with the ATLAS Collaboration at CERN. I am very excited for the opportunity to apply my skillset to disease outbreak modelling and its connection to the decision-making process, in particular after the COVID-19 pandemic, which brought the importance of this field to a new light.

My main research topic has been the measurement of the interaction of the Higgs boson with the W boson. For this, I have participated in several ATLAS analyses, working in all steps of the analysis pipeline, from defining optimal (rare) signal identification and event selection criteria, quantifying the impact of different sources of uncertainty and developing the statistical model. This resulted in two publications: \href{https://doi.org/10.1016/j.physletb.2021.136204}{Physics Letters B 816 (2021) 136204} and \href{https://arxiv.org/abs/2410.19611}{arXiv:2410.19611}. My research has further focused on the search for new physics in this interaction, related to the observed matter-antimatter asymmetry in the Universe. For this, I proposed and am now leading a novel ATLAS search, working on all steps of the analysis pipeline as well as supervising a master student.

These analyses use likelihood-based statistical inference procedures, but estimate the likelihoods from histograms of a small number of variables, which leads to suboptimal precision. To circumvent this limitation, I explored the use of a machine learning simulation-based inference (also called likelihood-free inference) method that estimates a sufficient statistic of the likelihood in a per-event basis with a large number of input variables. This work resulted in a publication - \href{http://dx.doi.org/10.1007/JHEP04(2024)014}{JHEP04(2024)014} (of which I was the main author) showing that such a method has a performance competitive with the traditional approach, while being robust to background processes (noise) and unobservable variables. I am now supervising a master student on the extension of that work, exploring a simulation-based inference method to estimate likelihood ratios using parametrized neural networks.

With my experience in data analysis, machine learning and collaborative work practices, I am confident that I would be an excellent fit for this position. I am eager to explore the use likelihood-free inference techniques in disease modelling, combining the knowledge in mechanistic models with machine learning techniques, all while allowing proper uncertainty quantification, key for the use of models in the decision-making process. I am looking forward to the opportunity to discuss further how my research experience and skills align with the requirements of the postdoctoral position. Thank you for your time and consideration.

\end{cvletter}


%-------------------------------------------------------------------------------
% Print the signature and enclosures with above letter information
\makeletterclosing

\end{document}
