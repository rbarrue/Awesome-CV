%!TEX TS-program = xelatex
%!TEX encoding = UTF-8 Unicode
% Awesome CV LaTeX Template for Cover Letter
%
% This template has been downloaded from:
% https://github.com/posquit0/Awesome-CV
%
% Authors:
% Claud D. Park <posquit0.bj@gmail.com>
% Lars Richter <mail@ayeks.de>
%
% Template license:
% CC BY-SA 4.0 (https://creativecommons.org/licenses/by-sa/4.0/)
%


%-------------------------------------------------------------------------------
% CONFIGURATIONS
%-------------------------------------------------------------------------------
% A4 paper size by default, use 'letterpaper' for US letter
\documentclass[11pt, a4paper]{awesome-cv}

% Configure page margins with geometry
\geometry{left=1.4cm, top=.8cm, right=1.4cm, bottom=1.8cm, footskip=.5cm}

% Color for highlights
% Awesome Colors: awesome-emerald, awesome-skyblue, awesome-red, awesome-pink, awesome-orange
%                 awesome-nephritis, awesome-concrete, awesome-darknight
\colorlet{awesome}{awesome-skyblue}
% Uncomment if you would like to specify your own color
% \definecolor{awesome}{HTML}{CA63A8}

% Colors for text
% Uncomment if you would like to specify your own color
% \definecolor{darktext}{HTML}{414141}
% \definecolor{text}{HTML}{333333}
% \definecolor{graytext}{HTML}{5D5D5D}
% \definecolor{lighttext}{HTML}{999999}
% \definecolor{sectiondivider}{HTML}{5D5D5D}

% Set false if you don't want to highlight section with awesome color
\setbool{acvSectionColorHighlight}{true}

% If you would like to change the social information separator from a pipe (|) to something else
\renewcommand{\acvHeaderSocialSep}{\quad\textbar\quad}


%-------------------------------------------------------------------------------
%	PERSONAL INFORMATION
%	Comment any of the lines below if they are not required
%-------------------------------------------------------------------------------
% Available options: circle|rectangle,edge/noedge,left/right
% \photo[circle,noedge,left]{./examples/profile}
\name{Ricardo}{Barrué}
\position{PhD researcher at \href{https://www.lip.pt/?lang=en}{Laboratório de Instrumentação e Física Experimental de Partículas (LIP)} and \href{https://tecnico.ulisboa.pt/en/}{Instituto Superior Técnico (IST)}}
\address{LIP: Av. Prof. Gama Pinto, n.2, 1649-003 Lisboa, Portugal; IST: Av. Rovisco Pais, 1, 1049-001 Lisboa, Portugal}

\email{ricardo.barrue@cern.ch}
%\dateofbirth{January 1st, 1970}
% \homepage{www.posquit0.com}
\github{rbarrue}
\linkedin{ricardo-barrue}
\gitlab{rcoelhob}
\orcid{0000-0001-8985-5379}
% \gitlab{gitlab-id}
% \stackoverflow{SO-id}{SO-name}
% \twitter{@twit}
% \skype{skype-id}
% \reddit{reddit-id}
% \medium{madium-id}
% \kaggle{kaggle-id}
% \hackerrank{hackerrank-id}
% \googlescholar{googlescholar-id}{name-to-display}
%% \firstname and \lastname will be used
% \googlescholar{googlescholar-id}{}
% \extrainfo{extra information}

% \quote{``Be the change that you want to see in the world."}


%-------------------------------------------------------------------------------
%	LETTER INFORMATION
%	All of the below lines must be filled out
%-------------------------------------------------------------------------------
% The company being applied to
\recipient
  {Wouter Verkerke, Clara Nellist, Flavia de Almeida Dias}
  {NIKHEF \\ Science Park 105 \\ 1098 XG Amsterdam} 
% The date on the letter, default is the date of compilation
\letterdate{\today}
% The title of the letter
\lettertitle{Job Application for Postdoc Position in the ATLAS Group}
% How the letter is opened
\letteropening{Dear Prof. Verkerke, Dr. Clara Nellist, Dr. Flavia de Almeida Dias}
% How the letter is closed
\letterclosing{Sincerely,}
% Any enclosures with the letter
\letterenclosure[Attached]{Curriculum Vitae}


%-------------------------------------------------------------------------------
\begin{document}

% Print the header with above personal information
% Give optional argument to change alignment(C: center, L: left, R: right)
\makecvheader[R]

% Print the footer with 3 arguments(<left>, <center>, <right>)
% Leave any of these blank if they are not needed
\makecvfooter
  {\today}
  {Ricardo Barrué ~~~·~~~ Cover Letter}
  {}

% Print the title with above letter information
\makelettertitle

%-------------------------------------------------------------------------------
%	LETTER CONTENT
%-------------------------------------------------------------------------------
\begin{cvletter}

  As a PhD researcher at LIP/IST working with the ATLAS experiment, I have developed extensive expertise in Higgs boson measurements. I proposed and led the first ATLAS search for CP-violating BSM components in the HWW vertex in $W(\to \ell \nu)H(\to b\bar{b})$ production, building on my previous contributions to VH measurements. These include the combined $VH(\to b\bar{b}/c\bar{c})$ measurement, where I worked on high-dimensional reweighting techniques using neural networks for shape uncertainty estimation. 

  My research has also focused on using neural simulation-based inference (NSBI) methods to improve analyses beyond traditional histogram-based approaches. I was the main author of a paper demonstrating how estimators of sufficient statistics of the likelihood can improve searches for CP violation in $W(\to \ell \nu)H(\to b\bar{b})$ production. Additionally, I have supervised a student on the follow-up of that work, exploring per-event (unbinned) parametrized estimators of the likelihood ratio to search for new physics contributions in the same process.

  My experience with $VH$ production provides a strong foundation for di-Higgs analyses (in particular those with $H\to b\bar{b}$ final states). I understand the specific difficulties of di-Higgs searches, such as the very low cross-sections and the complex backgrounds which are quite difficult to model in simulation. Looking ahead, I see some promising research directions to improve their sensitivity:
  \begin{itemize}
    \item Exploring techniques such as Optimal Transport for data-driven background estimations
    \item Explore simultaneous $b-$ and $\tau-$tagging with GN2
    \item Incorporating data-driven background estimations into NSBI methods
  \end{itemize}

  While my expertise is mainly with traditional ML applications, I am eager to expand into more advanced methods such as GNNs and transformers to pursue these directions. I also bring valuable experience in student supervision and coordination, which aligns well with the emphasis of this position on coordinating PhD student work.
  
  I am adaptable to any research direction the group wants to explore, and I am genuinely excited about the prospect of joining NIKHEF's ATLAS group. I look forward to discussing how my skills, experience, and research interests can support your research goals.

\end{cvletter}


%-------------------------------------------------------------------------------
% Print the signature and enclosures with above letter information
\makeletterclosing

\end{document}
