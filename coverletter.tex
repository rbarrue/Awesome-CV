%!TEX TS-program = xelatex
%!TEX encoding = UTF-8 Unicode
% Awesome CV LaTeX Template for Cover Letter
%
% This template has been downloaded from:
% https://github.com/posquit0/Awesome-CV
%
% Authors:
% Claud D. Park <posquit0.bj@gmail.com>
% Lars Richter <mail@ayeks.de>
%
% Template license:
% CC BY-SA 4.0 (https://creativecommons.org/licenses/by-sa/4.0/)
%


%-------------------------------------------------------------------------------
% CONFIGURATIONS
%-------------------------------------------------------------------------------
% A4 paper size by default, use 'letterpaper' for US letter
\documentclass[11pt, a4paper]{awesome-cv}

% Configure page margins with geometry
\geometry{left=1.4cm, top=.8cm, right=1.4cm, bottom=1.8cm, footskip=.5cm}

% Color for highlights
% Awesome Colors: awesome-emerald, awesome-skyblue, awesome-red, awesome-pink, awesome-orange
%                 awesome-nephritis, awesome-concrete, awesome-darknight
\colorlet{awesome}{awesome-skyblue}
% Uncomment if you would like to specify your own color
% \definecolor{awesome}{HTML}{CA63A8}

% Colors for text
% Uncomment if you would like to specify your own color
% \definecolor{darktext}{HTML}{414141}
% \definecolor{text}{HTML}{333333}
% \definecolor{graytext}{HTML}{5D5D5D}
% \definecolor{lighttext}{HTML}{999999}
% \definecolor{sectiondivider}{HTML}{5D5D5D}

% Set false if you don't want to highlight section with awesome color
\setbool{acvSectionColorHighlight}{true}

% If you would like to change the social information separator from a pipe (|) to something else
\renewcommand{\acvHeaderSocialSep}{\quad\textbar\quad}


%-------------------------------------------------------------------------------
%	PERSONAL INFORMATION
%	Comment any of the lines below if they are not required
%-------------------------------------------------------------------------------
% Available options: circle|rectangle,edge/noedge,left/right
% \photo[circle,noedge,left]{./examples/profile}
\name{Ricardo}{Barrué}
\position{PhD researcher at \href{https://www.lip.pt/?lang=en}{Laboratório de Instrumentação e Física Experimental de Partículas (LIP)} and \href{https://tecnico.ulisboa.pt/en/}{Instituto Superior Técnico (IST)}}
\address{LIP: Av. Prof. Gama Pinto, n.2, 1649-003 Lisboa, Portugal; IST: Av. Rovisco Pais, 1, 1049-001 Lisboa, Portugal}

\email{ricardo.barrue@cern.ch}
%\dateofbirth{January 1st, 1970}
% \homepage{www.posquit0.com}
\github{rbarrue}
\linkedin{ricardo-barrue}
\gitlab{rcoelhob}
\orcid{0000-0001-8985-5379}
% \gitlab{gitlab-id}
% \stackoverflow{SO-id}{SO-name}
% \twitter{@twit}
% \skype{skype-id}
% \reddit{reddit-id}
% \medium{madium-id}
% \kaggle{kaggle-id}
% \hackerrank{hackerrank-id}
% \googlescholar{googlescholar-id}{name-to-display}
%% \firstname and \lastname will be used
% \googlescholar{googlescholar-id}{}
% \extrainfo{extra information}

% \quote{``Be the change that you want to see in the world."}


%-------------------------------------------------------------------------------
%	LETTER INFORMATION
%	All of the below lines must be filled out
%-------------------------------------------------------------------------------
% The company being applied to
\recipient
  {Hans-Christian Schultz-Coulon, Rainer Stamen}
  {Kirchhoff-Institute for Physics \\ Im Neuenheimer Feld 227  \\ 69120 Heidelberg  } 
% The date on the letter, default is the date of compilation
\letterdate{\today}
% The title of the letter
\lettertitle{Job Application for Postdoc Position in the ATLAS Group}
% How the letter is opened
\letteropening{Dear Prof. Schultz-Coulon, Dr. Stamen}
% How the letter is closed
\letterclosing{Sincerely,}
% Any enclosures with the letter
% \letterenclosure[Attached]{Curriculum Vitae}


%-------------------------------------------------------------------------------
\begin{document}

% Print the header with above personal information
% Give optional argument to change alignment(C: center, L: left, R: right)
\makecvheader[R]

% Print the footer with 3 arguments(<left>, <center>, <right>)
% Leave any of these blank if they are not needed
\makecvfooter
  {\today}
  {Ricardo Barrué ~~~·~~~ Research Statement}
  {}

% Print the title with above letter information
\makelettertitle

%-------------------------------------------------------------------------------
%	LETTER CONTENT
%-------------------------------------------------------------------------------
\begin{cvletter}

As a PhD researcher in the ATLAS group at LIP/IST, I worked extensively with the jet trigger group and was a key element in the validation of its upgrades for Run 3, namely the introduction of full event scan tracking and Particle Flow jet reconstruction at the HLT. I carried out several performance studies that helped defined the optimal reconstruction parameters and also developed and maintained the relevant software. Additionally, during my time at CERN, I was part of the very small team of specialized trigger online on-call experts, critical to ATLAS data-taking. There, I acquired extensive knowledge of the ATLAS detector and its trigger system and the challenges of operating such systems in continuously changing conditions, developing excellent communication skills through constant coordination with ATLAS Run Coordination and various detector subsystems.

On the physics side, I have developed extensive expertise in Higgs boson measurements. I proposed and led the first ATLAS search for CP-violating components in $WH(\to b\bar{b})$ production, building on my key contributions to the combined $VH(\to b\bar{b}/c\bar{c})$ analysis and the first ATLAS search for boosted $VH(\to b\bar{b})$. I have also explored neural simulation-based inference (NSBI) methods to improve analyses beyond traditional histogram-based approaches, including as main author of a paper exploring such methods to improve the search for CP violation in $WH(\to b\bar{b})$ production.

In order to maximize the physics potential in the harsh conditions of HL-LHC operation, ATLAS will undergo extensive upgrades to its detector and trigger systems. I am interested in contributing to this effort, in particular in the development of trigger-related systems, in particular those using systems such as FPGAs. While I do not have experience with FPGA development, I believe my expertise with the ATLAS trigger system is a solid basis for developing such a skill. Additionally, I am interested in improving the precision of measurements of SM phenomena which may provide hints of new physics. I am particularly interested in bringing NSBI methods to ATLAS measurements of electroweak phenomena, where they have never been explored.
  
Despite the research directions pointed here, I am curious, a fast learner and adaptable to any research direction the group wants to explore. I am genuinely excited about the prospect of joining KIP's ATLAS group and look forward to discussing how my skills, experience, and interests can support your research goals.

\end{cvletter}


%-------------------------------------------------------------------------------
% Print the signature and enclosures with above letter information
\makeletterclosing

\end{document}
