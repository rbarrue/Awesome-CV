%!TEX TS-program = xelatex
%!TEX encoding = UTF-8 Unicode
% Awesome CV LaTeX Template for Cover Letter
%
% This template has been downloaded from:
% https://github.com/posquit0/Awesome-CV
%
% Authors:
% Claud D. Park <posquit0.bj@gmail.com>
% Lars Richter <mail@ayeks.de>
%
% Template license:
% CC BY-SA 4.0 (https://creativecommons.org/licenses/by-sa/4.0/)
%


%-------------------------------------------------------------------------------
% CONFIGURATIONS
%-------------------------------------------------------------------------------
% A4 paper size by default, use 'letterpaper' for US letter
\documentclass[11pt, a4paper]{awesome-cv}

% Configure page margins with geometry
\geometry{left=1.4cm, top=.8cm, right=1.4cm, bottom=1.8cm, footskip=.5cm}

% Color for highlights
% Awesome Colors: awesome-emerald, awesome-skyblue, awesome-red, awesome-pink, awesome-orange
%                 awesome-nephritis, awesome-concrete, awesome-darknight
\colorlet{awesome}{awesome-red}
% Uncomment if you would like to specify your own color
% \definecolor{awesome}{HTML}{CA63A8}

% Colors for text
% Uncomment if you would like to specify your own color
% \definecolor{darktext}{HTML}{414141}
% \definecolor{text}{HTML}{333333}
% \definecolor{graytext}{HTML}{5D5D5D}
% \definecolor{lighttext}{HTML}{999999}
% \definecolor{sectiondivider}{HTML}{5D5D5D}

% Set false if you don't want to highlight section with awesome color
\setbool{acvSectionColorHighlight}{true}

% If you would like to change the social information separator from a pipe (|) to something else
\renewcommand{\acvHeaderSocialSep}{\quad\textbar\quad}


%-------------------------------------------------------------------------------
%	PERSONAL INFORMATION
%	Comment any of the lines below if they are not required
%-------------------------------------------------------------------------------
% Available options: circle|rectangle,edge/noedge,left/right
% \photo[circle,noedge,left]{./examples/profile}
\name{Ricardo}{Barrué}
\position{PhD researcher at \href{https://www.lip.pt/?lang=en}{Laboratório de Instrumentação e Física Experimental de Partículas (LIP)} and \href{https://tecnico.ulisboa.pt/en/}{Instituto Superior Técnico (IST)}}
\address{LIP: Av. Prof. Gama Pinto, n.2, 1649-003 Lisboa, Portugal; IST: Av. Rovisco Pais, 1, 1049-001 Lisboa, Portugal}

\email{ricardo.barrue@cern.ch}
%\dateofbirth{January 1st, 1970}
% \homepage{www.posquit0.com}
\github{rbarrue}
\linkedin{ricardo-barrue}
\gitlab{rcoelhob}
\orcid{0000-0001-8985-5379}
% \gitlab{gitlab-id}
% \stackoverflow{SO-id}{SO-name}
% \twitter{@twit}
% \skype{skype-id}
% \reddit{reddit-id}
% \medium{madium-id}
% \kaggle{kaggle-id}
% \hackerrank{hackerrank-id}
% \googlescholar{googlescholar-id}{name-to-display}
%% \firstname and \lastname will be used
% \googlescholar{googlescholar-id}{}
% \extrainfo{extra information}

% \quote{``Be the change that you want to see in the world."}


%-------------------------------------------------------------------------------
%	LETTER INFORMATION
%	All of the below lines must be filled out
%-------------------------------------------------------------------------------
% The company being applied to
\recipient
  {Kyle Cranmer}
  {University of Wisconsin-Madison, Data Science Institute \\ 447 Lorch St. \\ Madison, WI 53706}
% The date on the letter, default is the date of compilation
\letterdate{\today}
% The title of the letter
\lettertitle{Job Application for Postdoctoral Research Associate}
% How the letter is opened
\letteropening{Dear Mr. Cranmer,}
% How the letter is closed
\letterclosing{Sincerely,}
% Any enclosures with the letter
\letterenclosure[Attached]{Curriculum Vitae}


%-------------------------------------------------------------------------------
\begin{document}

% Print the header with above personal information
% Give optional argument to change alignment(C: center, L: left, R: right)
\makecvheader[R]

% Print the footer with 3 arguments(<left>, <center>, <right>)
% Leave any of these blank if they are not needed
\makecvfooter
  {\today}
  {Ricardo Barrué ~~~·~~~ Cover Letter}
  {}

% Print the title with above letter information
\makelettertitle

%-------------------------------------------------------------------------------
%	LETTER CONTENT
%-------------------------------------------------------------------------------
\begin{cvletter}

I am writing to express my interest in applying for the postdoctoral position in the Cranmer group. I am finishing my PhD at the University of Lisbon with the thesis "Study of the Spin/CP properties of the Higgs coupling to W bosons with ATLAS at the LHC" (exp. finish date Dec. 2024). My main research topic has been the interactions of the Higgs boson with the W boson measured in associated WH production. For this, I have participated in two ATLAS analyses of this process. My research has further focused on the search for BSM sources of CP violation in this interaction and production channel. 

In this vein, I am now pushing for the first ATLAS (possibly LHC) search for CP violation in WH production using histograms of angular observables, an approach studied extensively in the phenomenological literature. A significant problem is that these angular observables require explicit neutrino reconstruction, which has significant systematic uncertainties. 

During my research, I explored a neural simulation-based inference method defined in \href{http://dx.doi.org/10.1073/pnas.1915980117}{arXiv:1805.12244}, called SALLY, which estimates a detector-level optimal observable to check if optimal performance could be obtained without explicit neutrino reconstruction. This exploration led to a publication in JHEP (\href{http://dx.doi.org/10.1007/JHEP04(2024)014}{JHEP04(2024)014} - of which I was the leading analyser), where we showed that a SALLY observable has a performance competitive with the traditional approach without requiring neutrino reconstruction. I am now supervising a master student on the extension of that work, with a method that reconstructs the likelihood ratio using parametrized neural networks - ALICE(S), which we expect to have superior performance to SALLY.

With my experience in physics analysis and simulation-based inference methods and my competence in collaborative software development, I am confident that I would be an excellent fit for this position. I am eager to apply simulation-based inference techniques to future ATLAS analyses, particularly for constraints on EFT coefficients and the trilinear Higgs coupling. Additionally, I am excited about the prospect of including simulation-based inference techniques in the pipeline of the Analysis Grand Challenge, developing new tooling wherever necessary to allow the easy use of these methods in HL-LHC analyses. I am looking forward to the opportunity to discuss further how my research experience and skills align with the requirements of the postdoctoral position. Thank you for your time and consideration.

\end{cvletter}


%-------------------------------------------------------------------------------
% Print the signature and enclosures with above letter information
\makeletterclosing

\end{document}
