%!TEX TS-program = xelatex
%!TEX encoding = UTF-8 Unicode
% Awesome CV LaTeX Template for Cover Letter
%
% This template has been downloaded from:
% https://github.com/posquit0/Awesome-CV
%
% Authors:
% Claud D. Park <posquit0.bj@gmail.com>
% Lars Richter <mail@ayeks.de>
%
% Template license:
% CC BY-SA 4.0 (https://creativecommons.org/licenses/by-sa/4.0/)
%


%-------------------------------------------------------------------------------
% CONFIGURATIONS
%-------------------------------------------------------------------------------
% A4 paper size by default, use 'letterpaper' for US letter
\documentclass[11pt, a4paper]{awesome-cv}

% Configure page margins with geometry
\geometry{left=1.4cm, top=.8cm, right=1.4cm, bottom=1.8cm, footskip=.5cm}

% Color for highlights
% Awesome Colors: awesome-emerald, awesome-skyblue, awesome-red, awesome-pink, awesome-orange
%                 awesome-nephritis, awesome-concrete, awesome-darknight
\colorlet{awesome}{awesome-skyblue}
% Uncomment if you would like to specify your own color
% \definecolor{awesome}{HTML}{CA63A8}

% Colors for text
% Uncomment if you would like to specify your own color
% \definecolor{darktext}{HTML}{414141}
% \definecolor{text}{HTML}{333333}
% \definecolor{graytext}{HTML}{5D5D5D}
% \definecolor{lighttext}{HTML}{999999}
% \definecolor{sectiondivider}{HTML}{5D5D5D}

% Set false if you don't want to highlight section with awesome color
\setbool{acvSectionColorHighlight}{true}

% If you would like to change the social information separator from a pipe (|) to something else
\renewcommand{\acvHeaderSocialSep}{\quad\textbar\quad}


%-------------------------------------------------------------------------------
%	PERSONAL INFORMATION
%	Comment any of the lines below if they are not required
%-------------------------------------------------------------------------------
% Available options: circle|rectangle,edge/noedge,left/right
% \photo[circle,noedge,left]{./examples/profile}
\name{Ricardo}{Barrué}
\position{PhD researcher at \href{https://www.lip.pt/}{Laboratório de Instrumentação e Física Experimental de Partículas (LIP)} and \href{https://tecnico.ulisboa.pt/en/}{Instituto Superior Técnico (IST)}}
\address{LIP: Av. Prof. Gama Pinto, n.2, 1649-003 Lisboa, Portugal; IST: Av. Rovisco Pais, 1, 1049-001 Lisboa, Portugal}

\email{ricardo.barrue@cern.ch}
%\dateofbirth{January 1st, 1970}
% \homepage{www.posquit0.com}
\github{rbarrue}
\linkedin{ricardo-barrue}
\gitlab{rcoelhob}
\orcid{0000-0001-8985-5379}
% \gitlab{gitlab-id}
% \stackoverflow{SO-id}{SO-name}
% \twitter{@twit}
% \skype{skype-id}
% \reddit{reddit-id}
% \medium{madium-id}
% \kaggle{kaggle-id}
% \hackerrank{hackerrank-id}
% \googlescholar{googlescholar-id}{name-to-display}
%% \firstname and \lastname will be used
% \googlescholar{googlescholar-id}{}
% \extrainfo{extra information}

% \quote{``Be the change that you want to see in the world."}


%-------------------------------------------------------------------------------
%	LETTER INFORMATION
%	All of the below lines must be filled out
%-------------------------------------------------------------------------------
% The company being applied to
\recipient
  {DESY Fellowship Commitee}
  {Deutsches Elektronen-Synchrotron DESY \\ Notkestraße 85 \\ 22607 Hamburg}
% The date on the letter, default is the date of compilation
\letterdate{\today}
% The title of the letter
\lettertitle{Job Application for DESY-Fellowship in Experimental Particle Physics}
% How the letter is opened
\letteropening{Dear Committee,}
% How the letter is closed
\letterclosing{Sincerely,}
% Any enclosures with the letter
% \letterenclosure[Attached]{Curriculum Vitae}


%-------------------------------------------------------------------------------
\begin{document}

% Print the header with above personal information
% Give optional argument to change alignment(C: center, L: left, R: right)
\makecvheader[R]

% Print the footer with 3 arguments(<left>, <center>, <right>)
% Leave any of these blank if they are not needed
\makecvfooter
  {\today}
  {Ricardo Barrué ~~~·~~~ Motivation Letter}
  {}

% Print the title with above letter information
\makelettertitle

%-------------------------------------------------------------------------------
%	LETTER CONTENT
%-------------------------------------------------------------------------------
\begin{cvletter}

I am writing to express my strong interest in the DESY-Fellowship in Experimental Particle Physics. This is a unique opportunity to work at one of the leading laboratories in the field and aligns perfectly with my research and career goals.

There are deep questions for which the Standard Model (SM) does not provide an answer, such as the observed value of the baryonic asymmetric of the Universe or the origin of electroweak symmetry breaking. I believe that one of the most effective ways to answer such questions is via very precise measurements of SM processes, where new physics effects may lead to deviations. Furthermore, processes with large cross-sections and available datasets are an excellent "playgrounds" to explore machine-learning-based techniques such as multi-dimensional unfolding (e.g. Omnifold) or simulation-based inference (SBI), both of which I am very keen to explore.

The High-Luminosity LHC will have a dramatic increase in the collision rate and number of simultaneous proton-proton interactions (from 65 in Run 3 to 200). The experiments will require significant upgrades to their trigger and data acquisition systems, in order to maximize their physics potential in the face of such harsh conditions. In this sector, I am interested in exploring the implementation of machine-learning algorithms in hardware triggers and in the development of software for trigger and data acquisition pipelines.

My background has prepared me well for such research. I am currently leading the first ATLAS search for CP-violating BSM components in the HWW vertex in $W(\to \ell \nu)H(\to b\bar{b})$ production, building on my contribution to the latest ATLAS VH production measurement (\href{https://arxiv.org/abs/2410.19611}{arXiv:2410.19611}). I have hands-on experience with the application of neural-network-based SBI methods, as the main author of a paper applying them to probe the interaction between the W and Higgs bosons, published in \href{http://dx.doi.org/10.1007/JHEP04(2024)014}{JHEP04(2024)014}. I also have substantial experience with trigger systems, as I contributed to the upgrade of the ATLAS Run 3 jet-based software trigger and served as trigger online expert on-call. During my research, I also developed experience in collaborative work, research strategy definition, supervision of students and science communication.

My skills and experience, combined with my adaptability, enthusiasm and desire to learn, make me a strong candidate for this position (independent of work group). Thank you for your consideration, I look forward to discussing how I can contribute to DESY and its goals.

\end{cvletter}


%-------------------------------------------------------------------------------
% Print the signature and enclosures with above letter information
\makeletterclosing

\end{document}
