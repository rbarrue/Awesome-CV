%!TEX TS-program = xelatex
%!TEX encoding = UTF-8 Unicode
% Awesome CV LaTeX Template for Cover Letter
%
% This template has been downloaded from:
% https://github.com/posquit0/Awesome-CV
%
% Authors:
% Claud D. Park <posquit0.bj@gmail.com>
% Lars Richter <mail@ayeks.de>
%
% Template license:
% CC BY-SA 4.0 (https://creativecommons.org/licenses/by-sa/4.0/)
%


%-------------------------------------------------------------------------------
% CONFIGURATIONS
%-------------------------------------------------------------------------------
% A4 paper size by default, use 'letterpaper' for US letter
\documentclass[11pt, a4paper]{awesome-cv}

% Configure page margins with geometry
\geometry{left=1.4cm, top=.8cm, right=1.4cm, bottom=1.8cm, footskip=.5cm}

% Color for highlights
% Awesome Colors: awesome-emerald, awesome-skyblue, awesome-red, awesome-pink, awesome-orange
%                 awesome-nephritis, awesome-concrete, awesome-darknight
\colorlet{awesome}{awesome-skyblue}
% Uncomment if you would like to specify your own color
% \definecolor{awesome}{HTML}{CA63A8}

% Colors for text
% Uncomment if you would like to specify your own color
% \definecolor{darktext}{HTML}{414141}
% \definecolor{text}{HTML}{333333}
% \definecolor{graytext}{HTML}{5D5D5D}
% \definecolor{lighttext}{HTML}{999999}
% \definecolor{sectiondivider}{HTML}{5D5D5D}

% Set false if you don't want to highlight section with awesome color
\setbool{acvSectionColorHighlight}{true}

% If you would like to change the social information separator from a pipe (|) to something else
\renewcommand{\acvHeaderSocialSep}{\quad\textbar\quad}


%-------------------------------------------------------------------------------
%	PERSONAL INFORMATION
%	Comment any of the lines below if they are not required
%-------------------------------------------------------------------------------
% Available options: circle|rectangle,edge/noedge,left/right
% \photo[circle,noedge,left]{./examples/profile}
\name{Ricardo}{Barrué}
\position{PhD researcher at \href{https://www.lip.pt/}{Laboratório de Instrumentação e Física Experimental de Partículas (LIP)} and \href{https://tecnico.ulisboa.pt/en/}{Instituto Superior Técnico (IST)}}
\address{LIP: Av. Prof. Gama Pinto, n.2, 1649-003 Lisboa, Portugal; IST: Av. Rovisco Pais, 1, 1049-001 Lisboa, Portugal}

\email{ricardo.barrue@cern.ch}
%\dateofbirth{January 1st, 1970}
% \homepage{www.posquit0.com}
\github{rbarrue}
\linkedin{ricardo-barrue}
\gitlab{rcoelhob}
\orcid{0000-0001-8985-5379}
% \gitlab{gitlab-id}
% \stackoverflow{SO-id}{SO-name}
% \twitter{@twit}
% \skype{skype-id}
% \reddit{reddit-id}
% \medium{madium-id}
% \kaggle{kaggle-id}
% \hackerrank{hackerrank-id}
% \googlescholar{googlescholar-id}{name-to-display}
%% \firstname and \lastname will be used
% \googlescholar{googlescholar-id}{}
% \extrainfo{extra information}

% \quote{``Be the change that you want to see in the world."}


%-------------------------------------------------------------------------------
%	LETTER INFORMATION
%	All of the below lines must be filled out
%-------------------------------------------------------------------------------
% The company being applied to
\recipient
  {Robert Schöfbeck}
  {HEPHY - Institute of High Energy Physics \\ Nikolsdorfer Gasse 18 \\ 1050 Wien, Austria} 
% The date on the letter, default is the date of compilation
\letterdate{\today}
% The title of the letter
\lettertitle{Job Application for postdoc position in physics analysis with the CMS experiment}
% How the letter is opened
\letteropening{Dear Prof. Schöfbeck,}
% How the letter is closed
\letterclosing{Sincerely,}
% Any enclosures with the letter
% \letterenclosure[Attached]{Curriculum Vitae}


%-------------------------------------------------------------------------------
\begin{document}

% Print the header with above personal information
% Give optional argument to change alignment(C: center, L: left, R: right)
\makecvheader[R]

% Print the footer with 3 arguments(<left>, <center>, <right>)
% Leave any of these blank if they are not needed
\makecvfooter
  {\today}
  {Ricardo Barrué ~~~·~~~ Motivation Letter}
  {}

% Print the title with above letter information
\makelettertitle

%-------------------------------------------------------------------------------
%	LETTER CONTENT
%-------------------------------------------------------------------------------
\begin{cvletter}

  I am writing to express my strong interest in the postdoc position for physics analysis with the CMS experiment at HEPHY. This opportunity to contribute to precision measurements and searches for new physics in the top quark sector aligns very well with my research experience and career goals.

  There are fundamental questions for which the Standard Model does not provide answers, such as the observed baryonic asymmetry of the Universe or the mechanism of electroweak symmetry breaking. One of the most effective approaches to address these questions is through precise measurements of SM processes, where effects from new physics may manifest as tiny deviations. My background has prepared me well for this research direction, having contributed to the ATLAS precision measurement of $V(\to \textrm{leptons})H(\to b\bar{b})$ production (\href{https://doi.org/10.1007/JHEP04(2025)075}{J. High Energ. Phys. 2025, 75 (2025)}). Additionally, I proposed and led ATLAS's first search for CP-violating EFT components in the HWW vertex in $W(\to \ell \nu)H(\to b\bar{b})$ events, published in \href{https://atlas.web.cern.ch/Atlas/GROUPS/PHYSICS/PUBNOTES/ATL-PHYS-PUB-2025-022/}{ATL-PHYS-PUB-2025-022} with world-leading constraints. 
  
  Traditional particle physics analyses rely on binned approaches that inevitably lead to information loss through the reduction of event kinematics to a small number of variables and histogramming. During my PhD, I have explored neural simulation-based inference methods to overcome these limitations and preserve the full kinematic information of the event without approximations. I was the main author of a paper on simulation-based inference methods to build detector-level optimal observables, applied to the same physics process (J. High Energ. Phys. 2024, 14).
  
  At HEPHY, I plan to focus on deploying and adapting unbinned methodologies for CMS analyses, exploring their performance with realistic systematic uncertainties while building expertise to tackle fundamental methodological challenges. The rich kinematic information in top quark decays presents an ideal testing ground for these techniques, with potential applications ranging from precision mass measurements to SMEFT operator constraints. Additionally, my experience with the ATLAS trigger system, including serving as trigger online expert on-call, would allow me to contribute to HEPHY's trigger development activities.

  My skills and experience, combined with my adaptability and enthusiasm for tackling complex methodological challenges, make me well-suited for this position. Thank you for your consideration. I look forward to discussing how I can contribute to HEPHY's research program.

\end{cvletter}


%-------------------------------------------------------------------------------
% Print the signature and enclosures with above letter information
\makeletterclosing

\end{document}
