%!TEX TS-program = xelatex
%!TEX encoding = UTF-8 Unicode
% Awesome CV LaTeX Template for Cover Letter
%
% This template has been downloaded from:
% https://github.com/posquit0/Awesome-CV
%
% Authors:
% Claud D. Park <posquit0.bj@gmail.com>
% Lars Richter <mail@ayeks.de>
%
% Template license:
% CC BY-SA 4.0 (https://creativecommons.org/licenses/by-sa/4.0/)
%


%-------------------------------------------------------------------------------
% CONFIGURATIONS
%-------------------------------------------------------------------------------
% A4 paper size by default, use 'letterpaper' for US letter
\documentclass[11pt, a4paper]{awesome-cv}

% Configure page margins with geometry
\geometry{left=1.4cm, top=.8cm, right=1.4cm, bottom=1.8cm, footskip=.5cm}

% Color for highlights
% Awesome Colors: awesome-emerald, awesome-skyblue, awesome-red, awesome-pink, awesome-orange
%                 awesome-nephritis, awesome-concrete, awesome-darknight
\colorlet{awesome}{awesome-skyblue}
% Uncomment if you would like to specify your own color
% \definecolor{awesome}{HTML}{CA63A8}

% Colors for text
% Uncomment if you would like to specify your own color
% \definecolor{darktext}{HTML}{414141}
% \definecolor{text}{HTML}{333333}
% \definecolor{graytext}{HTML}{5D5D5D}
% \definecolor{lighttext}{HTML}{999999}
% \definecolor{sectiondivider}{HTML}{5D5D5D}

% Set false if you don't want to highlight section with awesome color
\setbool{acvSectionColorHighlight}{true}

% If you would like to change the social information separator from a pipe (|) to something else
\renewcommand{\acvHeaderSocialSep}{\quad\textbar\quad}


%-------------------------------------------------------------------------------
%	PERSONAL INFORMATION
%	Comment any of the lines below if they are not required
%-------------------------------------------------------------------------------
% Available options: circle|rectangle,edge/noedge,left/right
% \photo[circle,noedge,left]{./examples/profile}
\name{Ricardo}{Barrué}
\position{PhD researcher at \href{https://www.lip.pt/}{Laboratório de Instrumentação e Física Experimental de Partículas (LIP)} and \href{https://tecnico.ulisboa.pt/en/}{Instituto Superior Técnico (IST)}}
\address{LIP: Av. Prof. Gama Pinto, n.2, 1649-003 Lisboa, Portugal; IST: Av. Rovisco Pais, 1, 1049-001 Lisboa, Portugal}

\email{ricardo.barrue@cern.ch}
%\dateofbirth{January 1st, 1970}
% \homepage{www.posquit0.com}
\github{rbarrue}
\linkedin{ricardo-barrue}
\gitlab{rcoelhob}
\orcid{0000-0001-8985-5379}
% \gitlab{gitlab-id}
% \stackoverflow{SO-id}{SO-name}
% \twitter{@twit}
% \skype{skype-id}
% \reddit{reddit-id}
% \medium{madium-id}
% \kaggle{kaggle-id}
% \hackerrank{hackerrank-id}
% \googlescholar{googlescholar-id}{name-to-display}
%% \firstname and \lastname will be used
% \googlescholar{googlescholar-id}{}
% \extrainfo{extra information}

% \quote{``Be the change that you want to see in the world."}


%-------------------------------------------------------------------------------
%	LETTER INFORMATION
%	All of the below lines must be filled out
%-------------------------------------------------------------------------------
% The company being applied to
\recipient
  {Michael Kagan, Ariel Schwartzman}
  {SLAC National Accelerator Laboratory \\ 2575 Sand Hill Road \\ Menlo Park, CA 94025-7015} 
% The date on the letter, default is the date of compilation
\letterdate{\today}
% The title of the letter
\lettertitle{Research Statement}
% How the letter is opened
\letteropening{}
% How the letter is closed
\letterclosing{Kind regards}
% Any enclosures with the letter
% \letterenclosure[Attached]{Curriculum Vitae}


%-------------------------------------------------------------------------------
\begin{document}

% Print the header with above personal information
% Give optional argument to change alignment(C: center, L: left, R: right)
\makecvheader[R]

% Print the footer with 3 arguments(<left>, <center>, <right>)
% Leave any of these blank if they are not needed
\makecvfooter
  {\today}
  {Ricardo Barrué ~~~·~~~ Research statement}
  {}

% Print the title with above letter information
\makelettertitle

%-------------------------------------------------------------------------------
%	LETTER CONTENT
%-------------------------------------------------------------------------------
\begin{cvletter}

  My research work has focused on the precise measurement of the properties of the Higgs boson, which enables the search for minor deviations from the Standard Model expectation that could point us to new physics. As a member of the ATLAS Collaboration, I have gained extensive experience across the entire analysis pipeline. Additionally, I became an expert in the jet reconstruction performance and operation of the ATLAS trigger system. Simultaneously, I explored simulation-based inference methods to overcome limitations in histogram-based analyses and improve searches for the new physics. In this research statement, I describe my research expertise in more detail and outline future directions I aim to explore at SLAC, framed by the challenges of the High Luminosity LHC.

  \lettersection{Data Analysis and Machine Learning for Higgs boson measurements}
  
  During my masters, I contributed to the first ATLAS search of $V(\to \textrm{leptons})H(\to b\bar{b})$ in the high-momentum regime, where we expect higher sensitivity to new physics contributions. I optimized the Higgs boson identification and event selection criteria, improving 4-5\% in analysis significance. Given the relevance of my contributions to this analysis, I was awarded exceptional ATLAS authorship. This analysis is published on \href{https://doi.org/10.1016/j.physletb.2021.136204}{Physics Letters B 816 (2021) 136204}.
  
  During my PhD, I contributed to the combined precision measurement of $V(\to \textrm{leptons})H(\to b\bar{b}/c\bar{c})$. In this analysis, priors for shape modeling uncertainties are derived via a `two-point' shape comparison between samples from nominal and alternative generators. Usually, the alternative sample has large statistical fluctuations compared to the nominal one (due to the much lower statistics). Given that the large statistical fluctuations may lead to overestimating these uncertainties, a key technique used is reweighting the nominal sample to the alternative sample. This technique reduces the statistical fluctuations while preserving relevant shape information and was achieved by high-dimensional likelihood ratio estimation using neural network-based classifiers (see CARL, \href{https://arxiv.org/abs/1506.02169}{arXiv:1506.02169}). I trained several such classifiers for the W+jets background in the high-momentum regime and quantified the impact of this uncertainty in the final result. I also worked extensively on fine-tuning the statistical (fit) model, exploring the need for degrees of freedom in several nuisance parameters (e.g. decorrelation in $p_T^W$). The analysis was published in \href{https://arxiv.org/abs/2410.19611}{arXiv:2410.19611}, and my contributions were crucial for the precision of the final result.
  
  Building on this foundation, I proposed and am now leading ATLAS's first search for CP-violating BSM components in the HWW vertex in $W(\to \ell \nu)H(\to b\bar{b})$ production. I contributed to the complete analysis pipeline, from the choice of CP-sensitive observable, categorization and modeling studies, and the development of the statistical model. Additionally, I supervise a master's student working on this analysis, parametrizing (STXS) signal strengths in bins of the CP-sensitive observable as a function of the Wilson coefficient of the EFT operator responsible for CP violation in this interaction.
  
  \lettersection{Neural Simulation-based Inference for searches of new physics in the Higgs sector}
  
  ATLAS analyses (such as the ones mentioned above) use frequentist likelihood-based inference. Given that the full likelihood is intractable, they estimate the likelihood from histograms of a few variables, leading to suboptimal results. During my PhD, I explored neural simulation-based (likelihood-free) inference methods to overcome this limitation. These methods use information from event generators and neural networks to estimate quantities relevant for statistical inference (such as the likelihood or likelihood ratios) on a high-dimensional, per-event basis.
  
  To improve searches for CP violation in Higgs boson interactions, I explored a neural simulation-based inference method that estimates sufficient statistics of the likelihood (score). This work, published in \href{https://doi.org/10.1007/JHEP04(2024)014}{J. High Energ. Phys. 2024, 14 (2024)} (of which I was the primary author) showed that such a method has performance competitive with traditional histogram-based approaches while being robust to the presence of backgrounds and (unobservable) neutrinos in the final state. I now supervise a master's student on the extension of that work, exploring estimators of the likelihood ratio using parametrized neural networks, with explicit dependence on the value of relevant Wilson coefficients (numerator), using ensembles of neural networks for proper uncertainty quantification.
  
  \lettersection{Future research directions}
  
  The High-Luminosity LHC will bring remarkable challenges to the optimal exploitation of its output: events will become more complex (with up to 200 simultaneous collisions), and data rates will increase by almost an order of magnitude, but investment in computational resources is expected to remain constrained. Drawing from my experience, I propose three key research directions that address HL-LHC challenges and align with SLAC's priorities:
  
  \textbf{Advancing Neural Simulation-Based Inference}\\
  As we move toward precision measurements at the HL-LHC, my work on neural simulation-based inference methods can be extended by exploring more powerful architectures that can use low-level variables while targeting high-dimensional parameter spaces. From my experience, I believe it is also relevant to implement techniques that mitigate the significant computational requirements of these methods by using, e.g., `mining gold' techniques, described in \href{https://arxiv.org/abs/1907.10621}{arXiv:1907.10621}. These can be applied in several ATLAS analyses, such as measurements of EFT coefficients in the Higgs sector or trilinear Higgs coupling in di-Higgs production.
  
  \textbf{Uncertainty quantification in Machine Learning methods}\\
  Machine learning methods used in HEP analyses are becoming more advanced and more used across all steps of the analysis pipeline (from reconstruction to inference). This trend will grow as we move to the HL-LHC era. I believe it is thus required to develop proper uncertainty quantification procedures, which consider not only the traditional sources of uncertainty (e.g. statistical, modeling, detector) but also epistemic (training-related) uncertainties, something which was also highlighted in the recent Snowmass process (\href{https://arxiv.org/abs/2208.03284}{arXiv:2208.03284}). Building on my experience, I am interested in pursuing this topic further, developing conceptual and software frameworks to facilitate the adoption of such procedures.
  
  \textbf{Machine Learning at the Level-1 trigger}\\
  Innovative trigger strategies are critical to maximize our scientific output with the HL-HLC data. Drawing from my extensive experience with the ATLAS trigger system, I want to develop low-latency ML solutions for the Level-1 trigger. In particular, when implemented at Level-1, anomaly detection methods give us sensitivity to rare event topologies without any bias from previous trigger selections. Implementing these methods within the Level-1 constraints and in a high pile-up environment is challenging but essential for exploiting HL-LHC's full physics potential. My hands-on experience with trigger operations gives me a practical understanding of the constraints and challenges in implementing such solutions.
  
  The research directions above reflect my current interests and expertise. Still, we will need innovative solutions across the entire analysis pipeline, and I am eager to contribute to their development at SLAC. Throughout my research, I have demonstrated my ability to tackle complex technical challenges, learning quickly and adapting to evolving technical challenges. I am enthusiastic about joining SLAC's research environment, where I could contribute my experience and grow into new approaches essential for extracting the maximum out of the HL-LHC data.

\end{cvletter}

%-------------------------------------------------------------------------------
% Print the signature and enclosures with above letter information
\makeletterclosing

\end{document}