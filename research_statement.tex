%!TEX TS-program = xelatex
%!TEX encoding = UTF-8 Unicode
% Awesome CV LaTeX Template for Cover Letter
%
% This template has been downloaded from:
% https://github.com/posquit0/Awesome-CV
%
% Authors:
% Claud D. Park <posquit0.bj@gmail.com>
% Lars Richter <mail@ayeks.de>
%
% Template license:
% CC BY-SA 4.0 (https://creativecommons.org/licenses/by-sa/4.0/)
%


%-------------------------------------------------------------------------------
% CONFIGURATIONS
%-------------------------------------------------------------------------------
% A4 paper size by default, use 'letterpaper' for US letter
\documentclass[11pt, a4paper]{awesome-cv}

% Configure page margins with geometry
\geometry{left=1.4cm, top=.8cm, right=1.4cm, bottom=1.8cm, footskip=.5cm}

% Color for highlights
% Awesome Colors: awesome-emerald, awesome-skyblue, awesome-red, awesome-pink, awesome-orange
%                 awesome-nephritis, awesome-concrete, awesome-darknight
\colorlet{awesome}{awesome-skyblue}
% Uncomment if you would like to specify your own color
% \definecolor{awesome}{HTML}{CA63A8}

% Colors for text
% Uncomment if you would like to specify your own color
% \definecolor{darktext}{HTML}{414141}
% \definecolor{text}{HTML}{333333}
% \definecolor{graytext}{HTML}{5D5D5D}
% \definecolor{lighttext}{HTML}{999999}
% \definecolor{sectiondivider}{HTML}{5D5D5D}

% Set false if you don't want to highlight section with awesome color
\setbool{acvSectionColorHighlight}{true}

% If you would like to change the social information separator from a pipe (|) to something else
\renewcommand{\acvHeaderSocialSep}{\quad\textbar\quad}


%-------------------------------------------------------------------------------
%	PERSONAL INFORMATION
%	Comment any of the lines below if they are not required
%-------------------------------------------------------------------------------
% Available options: circle|rectangle,edge/noedge,left/right
% \photo[circle,noedge,left]{./examples/profile}
\name{Ricardo}{Barrué}
\position{PhD researcher at \href{https://www.lip.pt/}{Laboratório de Instrumentação e Física Experimental de Partículas (LIP)} and \href{https://tecnico.ulisboa.pt/en/}{Instituto Superior Técnico (IST)}}
\address{LIP: Av. Prof. Gama Pinto, n.2, 1649-003 Lisboa, Portugal; IST: Av. Rovisco Pais, 1, 1049-001 Lisboa, Portugal}

\email{ricardo.barrue@cern.ch}
%\dateofbirth{January 1st, 1970}
% \homepage{www.posquit0.com}
\github{rbarrue}
\linkedin{ricardo-barrue}
\gitlab{rcoelhob}
\orcid{0000-0001-8985-5379}
% \gitlab{gitlab-id}
% \stackoverflow{SO-id}{SO-name}
% \twitter{@twit}
% \skype{skype-id}
% \reddit{reddit-id}
% \medium{madium-id}
% \kaggle{kaggle-id}
% \hackerrank{hackerrank-id}
% \googlescholar{googlescholar-id}{name-to-display}
%% \firstname and \lastname will be used
% \googlescholar{googlescholar-id}{}
% \extrainfo{extra information}

% \quote{``Be the change that you want to see in the world."}


%-------------------------------------------------------------------------------
%	LETTER INFORMATION
%	All of the below lines must be filled out
%-------------------------------------------------------------------------------
% The company being applied to
\recipient
  {Robert Schöfbeck}
  {HEPHY - Institute of High Energy Physics \\ Nikolsdorfer Gasse 18 \\ 1050 Wien, Austria} 
% The date on the letter, default is the date of compilation
\letterdate{\today}
% The title of the letter
\lettertitle{Research Statement}
% How the letter is opened
\letteropening{}
% How the letter is closed
\letterclosing{Kind regards}
% Any enclosures with the letter
% \letterenclosure[Attached]{Curriculum Vitae}


%-------------------------------------------------------------------------------
\begin{document}

% Print the header with above personal information
% Give optional argument to change alignment(C: center, L: left, R: right)
\makecvheader[R]

% Print the footer with 3 arguments(<left>, <center>, <right>)
% Leave any of these blank if they are not needed
\makecvfooter
  {\today}
  {Ricardo Barrué ~~~·~~~ Research statement}
  {}

% Print the title with above letter information
\makelettertitle

%-------------------------------------------------------------------------------
%	LETTER CONTENT
%-------------------------------------------------------------------------------
\begin{cvletter}

  My research work has focused on the precise measurement of the properties of the Higgs boson and the search for tiny deviations from the Standard Model expectation that may be a sign of new physics. As a member of the ATLAS Collaboration, I have gained extensive experience across the entire analysis pipeline. Additionally, I became an expert in the jet reconstruction performance and operation of the ATLAS trigger system. Simultaneously, I explored simulation-based inference methods to improve searches for the new physics. In this research statement, I describe my research expertise in more detail and outline future research directions I aim to explore at HEPHY, framed by open questions in the top quark sector.

  \lettersection{Data Analysis and Machine Learning for Higgs boson measurements}
  
  During my masters, I participated in the first ATLAS search of $V(\to \textrm{leptons})H(\to b\bar{b})$ in the high-momentum regime, where we expect higher sensitivity to new physics contributions. I optimized Higgs boson identification and event selection criteria and was awarded exceptional ATLAS authorship for this analysis, published in \href{https://doi.org/10.1016/j.physletb.2021.136204}{Physics Letters B 816 (2021) 136204}.
  
  During my PhD, I contributed to the precision measurement of $V(\to \textrm{leptons})H(\to b\bar{b})$ combined both low and high-momentum regimes. I worked on background modelling, including the training of a machine-learning-based reweighting method to reduce uncertainties from finite sample statistics, as well as in extensive fine-tuning of the statistical (fit) model. The analysis was published in \href{https://doi.org/10.1007/JHEP04(2025)075}{J. High Energ. Phys. 2025, 75 (2025)}, and my contributions were crucial for the precision of the final result.
  
  Building on the foundation of that analysis, I proposed and led ATLAS's first search for CP-violating EFT components in the HWW vertex in $W(\to \ell \nu)H(\to b\bar{b})$ production. I developed the complete analysis pipeline, from the choice of CP-sensitive observable, categorization and modelling studies, and the development of the statistical model. Additionally, I supervised a master student working on this analysis, parametrizing (STXS) signal strengths in bins of the CP-sensitive observable as a function of the Wilson coefficient of the EFT operator responsible for CP violation in this interaction. The analysis was published in \href{https://atlas.web.cern.ch/Atlas/GROUPS/PHYSICS/PUBNOTES/ATL-PHYS-PUB-2025-022/}{ATL-PHYS-PUB-2025-022}, containing the world's best constraints on the relevant Wilson coefficient.
  
  \lettersection{Neural Simulation-based Inference for searches of new physics in the Higgs sector}
  
  Particle physics analyses use frequentist inference procedures, with the likelihood function as the fundamental piece. Despite its importance, the calculation of the likelihood from the full event information is computationally unfeasible. Due to this, the likelihood has been approximated from a small number of physics-motivated variables, leading to suboptimal results due to loss of information. An alternative approach has used matrix element calculators to compress the full kinematic information of the event into an optimal summary statistic, but usually ignoring or approximating several steps in the simulation chain. Additionally, analyses have historically used binned density estimation methods such as histograms to estimate the likelihood from chosen observables, which leads to loss of information and a reduction in the sensitivity.

  During my PhD, I worked on neural simulation-based inference methods, unbinned methods which use neural networks and event generator information to estimate quantities relevant for statistical inference for each event without approximations or loss of sensitivity.
  
  I started by exploring a method to estimate sufficient statistics of the likelihood (score) without the need to ignore or approximate steps in the simulation chain (i.e. a detector-level optimal observable), applied to the search for CP-violating EFT components in the HWW interaction in $W(\to \ell \nu)H(\to b\bar{b})$ production. This work, published in \href{https://doi.org/10.1007/JHEP04(2024)014}{J. High Energ. Phys. 2024, 14 (2024)} (of which I was the primary author) showed that such a method has performance competitive with traditional kinematic and angular observables (such as those used in the ATLAS analysis), while being robust to the presence of backgrounds and (unobservable) neutrinos in the final state. I supervised a master student on the extension of that work, exploring unbinned estimators of the likelihood ratio that use parametrized neural networks to learn its dependence on the relevant Wilson coefficients (paper in preparation).
  
  \lettersection{Future research directions}

  While top quark properties have been precisely measured using binned approaches, the full kinematic information available in top quark decays remains underutilized. My expertise with Higgs analyses and experience with simulation-based inference methods has given me direct contact with the challenges of performing precision measurements with high-dimensional event data. Such knowledge is naturally transferable from the Higgs to the top quark sector, where unbinned approaches may significantly enhance both the precision of SM parameter measurements (such as the top mass) and sensitivity to deviations from SM predictions (such as those parametrized by SMEFT operators). Despite the increase in precision from unbinned techniques demonstrated in a few experimental analyses (e.g. \href{https://doi.org/10.1088/1361-6633/adcd9a}{Rep. Prog. Phys. 88 (2025) 05780} or \href{https://arxiv.org/abs/2505.17850}{arXiv:2505.17850}), there are still roadblocks to their widespread adoption, namely their novelty and complexity, as well as their delicate balance between performance and computational cost, in particular when taking into account a large number of systematic uncertainties.
    
  At HEPHY, I plan to focus initially on deploying and adapting existing unbinned methodologies for practical use in CMS analyses, exploring their performance with realistic systematic uncertainties and computational constraints. Building on recent developments, I envision several possibilities, such as particle-level unbinned measurements of energy correlators to extract the top quark mass (\href{https://doi.org/10.1007/JHEP04(2025)072}{J. High Energ. Phys. 2025, 72 (2025)}), or constraints on SMEFT operators in the top quark sector using simulation-based inference techniques (\href{https://doi.org/10.1088/2632-2153/ad9fd1}{Mach. Learn. Sci. Technol. 6 015007 (2025)}). Such an approach will also allow me to contribute immediately to analysis improvements while building the expertise needed to tackle the more fundamental methodological questions. 

  Beyond analysis work, my extensive experience with the performance and operation of the ATLAS trigger system will allow me to contribute to HEPHY's trigger development activities, providing a complementary skillset to my primary research focus.

  The research directions showcased above represent some of my ideas for this position. Nonetheless, throughout my research, I have demonstrated my ability to tackle complex challenges, learning quickly and adapting to evolving technical environments. I am enthusiastic about joining HEPHY's research environment, where I can contribute my experience and grow into new approaches for extracting the maximum out of current and future datasets.

\end{cvletter}

%-------------------------------------------------------------------------------
% Print the signature and enclosures with above letter information
\makeletterclosing

\end{document}