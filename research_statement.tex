%!TEX TS-program = xelatex
%!TEX encoding = UTF-8 Unicode
% Awesome CV LaTeX Template for Cover Letter
%
% This template has been downloaded from:
% https://github.com/posquit0/Awesome-CV
%
% Authors:
% Claud D. Park <posquit0.bj@gmail.com>
% Lars Richter <mail@ayeks.de>
%
% Template license:
% CC BY-SA 4.0 (https://creativecommons.org/licenses/by-sa/4.0/)
%


%-------------------------------------------------------------------------------
% CONFIGURATIONS
%-------------------------------------------------------------------------------
% A4 paper size by default, use 'letterpaper' for US letter
\documentclass[11pt, a4paper]{awesome-cv}

% Configure page margins with geometry
\geometry{left=1.4cm, top=.8cm, right=1.4cm, bottom=1.8cm, footskip=.5cm}

% Color for highlights
% Awesome Colors: awesome-emerald, awesome-skyblue, awesome-red, awesome-pink, awesome-orange
%                 awesome-nephritis, awesome-concrete, awesome-darknight
\colorlet{awesome}{awesome-skyblue}
% Uncomment if you would like to specify your own color
% \definecolor{awesome}{HTML}{CA63A8}

% Colors for text
% Uncomment if you would like to specify your own color
% \definecolor{darktext}{HTML}{414141}
% \definecolor{text}{HTML}{333333}
% \definecolor{graytext}{HTML}{5D5D5D}
% \definecolor{lighttext}{HTML}{999999}
% \definecolor{sectiondivider}{HTML}{5D5D5D}

% Set false if you don't want to highlight section with awesome color
\setbool{acvSectionColorHighlight}{true}

% If you would like to change the social information separator from a pipe (|) to something else
\renewcommand{\acvHeaderSocialSep}{\quad\textbar\quad}


%-------------------------------------------------------------------------------
%	PERSONAL INFORMATION
%	Comment any of the lines below if they are not required
%-------------------------------------------------------------------------------
% Available options: circle|rectangle,edge/noedge,left/right
% \photo[circle,noedge,left]{./examples/profile}
\name{Ricardo}{Barrué}
\position{PhD researcher at \href{https://www.lip.pt/}{Laboratório de Instrumentação e Física Experimental de Partículas (LIP)} and \href{https://tecnico.ulisboa.pt/en/}{Instituto Superior Técnico (IST)}}
\address{LIP: Av. Prof. Gama Pinto, n.2, 1649-003 Lisboa, Portugal; IST: Av. Rovisco Pais, 1, 1049-001 Lisboa, Portugal}

\email{ricardo.barrue@cern.ch}
%\dateofbirth{January 1st, 1970}
% \homepage{www.posquit0.com}
\github{rbarrue}
\linkedin{ricardo-barrue}
\gitlab{rcoelhob}
\orcid{0000-0001-8985-5379}
% \gitlab{gitlab-id}
% \stackoverflow{SO-id}{SO-name}
% \twitter{@twit}
% \skype{skype-id}
% \reddit{reddit-id}
% \medium{madium-id}
% \kaggle{kaggle-id}
% \hackerrank{hackerrank-id}
% \googlescholar{googlescholar-id}{name-to-display}
%% \firstname and \lastname will be used
% \googlescholar{googlescholar-id}{}
% \extrainfo{extra information}

% \quote{``Be the change that you want to see in the world."}


%-------------------------------------------------------------------------------
%	LETTER INFORMATION
%	All of the below lines must be filled out
%-------------------------------------------------------------------------------
% The company being applied to
\recipient
  {Marisa Eisenberg, Tom Schwarz}
  {University of Michigan/MICOM \\ 500 S. State Street \\ Ann Arbor, MI 48109} 
% The date on the letter, default is the date of compilation
\letterdate{\today}
% The title of the letter
\lettertitle{Research Statement}
% How the letter is opened
\letteropening{}
% How the letter is closed
\letterclosing{Kind regards}
% Any enclosures with the letter
% \letterenclosure[Attached]{Curriculum Vitae}


%-------------------------------------------------------------------------------
\begin{document}

% Print the header with above personal information
% Give optional argument to change alignment(C: center, L: left, R: right)
\makecvheader[R]

% Print the footer with 3 arguments(<left>, <center>, <right>)
% Leave any of these blank if they are not needed
\makecvfooter
  {\today}
  {Ricardo Barrué ~~~·~~~ Research statement}
  {}

% Print the title with above letter information
\makelettertitle

%-------------------------------------------------------------------------------
%	LETTER CONTENT
%-------------------------------------------------------------------------------
\begin{cvletter}

The core of my research has been the precise measurement of the properties of the Higgs boson, looking for tiny deviations from possible new physics that solves deep problems in our current theory. My research has given me vast experience in advanced data analysis - handling large volumes of complex data, searching for rare signal events amid large backgrounds, uncertainty quantification, and simulation-based inference methods - skills that can be adapted to tackle key challenges in epidemiological modeling.

\lettersection{Data Analysis and Uncertainty Quantification}

As an undergraduate student, I worked on a summer internship developing a search for Higgs boson pair production (a very rare process), using Boosted Decision Trees for signal-background separation. I proposed using of Principal Component Analysis and standardization of values on the input variable set, which, after retraining, increased  signal efficiency and background rejection. During my masters, I contributed to a novel ATLAS search for the production of high-energy Higgs bosons, a rare process where new physics models predict large deviations, an analysis published in \href{https://doi.org/10.1016/j.physletb.2021.136204}{Physics Letters B 816 (2021) 136204}. I worked on defining optimal signal identification and event selection criteria to maximize the signal-to-background ratio, an important step to improve the sensitivity of our search. My contributions to this analysis granted me exceptional authorship for this publication (granted to incoming ATLAS members who made significant contributions to a publication). The skills developed in both projects can be adapted to create a framework to detect early signs of disease outbreaks amid the baseline health data. % Connect to anomaly detection in future work

During my PhD, I contributed to a precision measurement of Higgs boson decays, where my main contribution was the derivation of simulation-based priors for several background uncertainties, to be used in a maximum-likelihood estimation procedure. A key technique used in this derivation was a neural network-based high-dimensional reweighting method to create proxies for low-statistics alternative samples (used in the uncertainty derivation procedure) from high-statistics nominal samples. This technique reduces the impact of statistical fluctuations that could lead to uncertainty overestimation and is critical to the precision of the measurement. As part of my contribution, I trained sets of these neural networks for different background processes. Such techniques can be employed to use data from previous outbreaks to inform the modeling of new ones, especially when the amount and quality of available information is reduced.

I also worked on development of the statistical model for several uncertainties, finding regions where they had different impacts on data and simulation (requiring additional degrees of freedom in the statistical model), and determining their impact on the uncertainty of the measurement. This analysis was published in \href{https://arxiv.org/abs/2410.19611}{arXiv:2410.19611}, and my contributions were crucial for the final result. The techniques acquired in this work are applicable to uncertainty quantification in outbreak modeling when combining simulation from mechanistic models and real data.

I proposed and am now leading a novel ATLAS search based on the precision measurement described above. I contributed to the complete analysis pipeline, from the choice of relevant observable to the study of its modeling in simulations and data, and the development of the statistical model, skills that translate very well into studies of outbreak modeling. Additionally, I also supervised a master's student working on this analysis.

\lettersection{Simulation-based inference methods}

The analyses mentioned above use frequentist likelihood-based inference and estimate the likelihoods from histograms of few variables (since the full likelihood is intractable), leading to suboptimal results. During my PhD, I explored simulation-based (also called likelihood-free) inference methods, which aim at overcoming this limitation. These methods use quality simulators (mechanistic or even deep generative models) and neural networks to estimate quantities relevant for statistical inference (such as the likelihood or likelihood ratios) on a high-dimensional, per-event basis.

To improve searches for new physics in Higgs boson interactions (mentioned in the introduction of this statement), I explored a simulation-based inference method that estimates sufficient statistics of the likelihood. This work, published in \href{http://dx.doi.org/10.1007/JHEP04(2024)014}{JHEP04(2024)014} (of which I was the primary author) showed that such a method has a performance competitive with the traditional approach while being robust to the presence of background processes and unobservable variables. I am now supervising a master's student on the extension of that work, exploring a related method to estimate likelihood ratios with parametrized neural networks (which we expect to have superior performance), using ensembles of neural networks for proper uncertainty quantification. 

These methods can be directly applied to epidemiology for parametric inference in mechanistic models, allowing proper uncertainty quantification through parameter sampling and simulation.

\lettersection{Future research directions}

Particle physics has long used hybrid methods combining high-quality mechanistic simulators with advanced machine learning. There is potential in further exploring similar methods in epidemiology, and I see several promising directions:

\textbf{Simulation-based inference:} recent advances in Bayesian formulations of simulation-based inference methods (e.g. \href{https://arxiv.org/abs/2306.16015}{\textit{BayesFlow: Amortized Bayesian Workflows With Neural Networks}}), coupled with improved epidemic simulators (driven by COVID-19) (e.g. \href{https://pubmed.ncbi.nlm.nih.gov/33958443/}{\textit{A kernel-modulated SIR model for Covid-19 contagious spread from county to continent}}), open new possibilities for robust parameter estimation with proper uncertainty quantification (e.g. \href{https://journals.plos.org/ploscompbiol/article?id=10.1371/journal.pcbi.1009472}{\textit{OutbreakFlow: Model-based Bayesian inference of disease outbreak dynamics with invertible neural networks (...)}}), which I want to further explore, given my existing knowledge of the topic.
  
\textbf{Differentiable simulators:} Recent work on differentiable mechanistic simulators allows direct parameter optimization using gradient descent when comparing predictions to observed data - a technique I'm eager to explore for outbreak modeling (e.g. \href{https://dl.acm.org/doi/abs/10.5555/3545946.3598851}{\textit{Differentiable agent-based epidemiology}}).

\textbf{Operational considerations:} Along with accurate models with proper uncertainty quantification procedures, there are additional constraints to take into account for the models to have real-world applicability and support the decision-making process:
\begin{itemize}
  \item computational efficiency, such that decision-makers can request and access predictions when needed
  \item robustness to missing data, common in real-life scenarios
  \item model explainability, to bridge the gap between the model output and decision-makers' knowledge and experience
\end{itemize}
One framework designed to solve these problems was published in \href{https://ojs.aaai.org/index.php/AAAI/article/view/17808}{\textit{DeepCOVID: An Operational Deep Learning-driven Framework for Explainable Real-time COVID-19 Forecasting}}, and I would like to support the development and improvement of this or similar tools.
  
\textbf{Communication:} I'm passionate and experienced in communicating science and technology to a non-expert audience. I want to use this skillset to bridge the gap between the output of complex methods and decision-makers through clear language and informative visualizations.

\end{cvletter}

%-------------------------------------------------------------------------------
% Print the signature and enclosures with above letter information
\makeletterclosing

\end{document}