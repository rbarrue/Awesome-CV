%!TEX TS-program = xelatex
%!TEX encoding = UTF-8 Unicode
% Awesome CV LaTeX Template for Cover Letter
%
% This template has been downloaded from:
% https://github.com/posquit0/Awesome-CV
%
% Authors:
% Claud D. Park <posquit0.bj@gmail.com>
% Lars Richter <mail@ayeks.de>
%
% Template license:
% CC BY-SA 4.0 (https://creativecommons.org/licenses/by-sa/4.0/)
%


%-------------------------------------------------------------------------------
% CONFIGURATIONS
%-------------------------------------------------------------------------------
% A4 paper size by default, use 'letterpaper' for US letter
\documentclass[11pt, a4paper]{awesome-cv}

% Configure page margins with geometry
\geometry{left=1.4cm, top=.8cm, right=1.4cm, bottom=1.8cm, footskip=.5cm}

% Color for highlights
% Awesome Colors: awesome-emerald, awesome-skyblue, awesome-red, awesome-pink, awesome-orange
%                 awesome-nephritis, awesome-concrete, awesome-darknight
\colorlet{awesome}{awesome-skyblue}
% Uncomment if you would like to specify your own color
% \definecolor{awesome}{HTML}{CA63A8}

% Colors for text
% Uncomment if you would like to specify your own color
% \definecolor{darktext}{HTML}{414141}
% \definecolor{text}{HTML}{333333}
% \definecolor{graytext}{HTML}{5D5D5D}
% \definecolor{lighttext}{HTML}{999999}
% \definecolor{sectiondivider}{HTML}{5D5D5D}

% Set false if you don't want to highlight section with awesome color
\setbool{acvSectionColorHighlight}{true}

% If you would like to change the social information separator from a pipe (|) to something else
\renewcommand{\acvHeaderSocialSep}{\quad\textbar\quad}


%-------------------------------------------------------------------------------
%	PERSONAL INFORMATION
%	Comment any of the lines below if they are not required
%-------------------------------------------------------------------------------
% Available options: circle|rectangle,edge/noedge,left/right
% \photo[circle,noedge,left]{./examples/profile}
\name{Ricardo}{Barrué}
\position{PhD researcher at \href{https://www.lip.pt/}{Laboratório de Instrumentação e Física Experimental de Partículas (LIP)} and \href{https://tecnico.ulisboa.pt/en/}{Instituto Superior Técnico (IST)}}
\address{LIP: Av. Prof. Gama Pinto, n.2, 1649-003 Lisboa, Portugal; IST: Av. Rovisco Pais, 1, 1049-001 Lisboa, Portugal}

\email{ricardo.barrue@cern.ch}
%\dateofbirth{January 1st, 1970}
% \homepage{www.posquit0.com}
\github{rbarrue}
\linkedin{ricardo-barrue}
\gitlab{rcoelhob}
\orcid{0000-0001-8985-5379}
% \gitlab{gitlab-id}
% \stackoverflow{SO-id}{SO-name}
% \twitter{@twit}
% \skype{skype-id}
% \reddit{reddit-id}
% \medium{madium-id}
% \kaggle{kaggle-id}
% \hackerrank{hackerrank-id}
% \googlescholar{googlescholar-id}{name-to-display}
%% \firstname and \lastname will be used
% \googlescholar{googlescholar-id}{}
% \extrainfo{extra information}

% \quote{``Be the change that you want to see in the world."}


%-------------------------------------------------------------------------------
%	LETTER INFORMATION
%	All of the below lines must be filled out
%-------------------------------------------------------------------------------
% The company being applied to
\recipient
  {Marisa Eisenberg, Tom Schwarz}
  {University of Michigan/MICOM \\ 500 S. State Street \\ Ann Arbor, MI 48109} 
% The date on the letter, default is the date of compilation
\letterdate{\today}
% The title of the letter
\lettertitle{Research Statement}
% How the letter is opened
\letteropening{}
% How the letter is closed
\letterclosing{Kind regards}
% Any enclosures with the letter
% \letterenclosure[Attached]{Curriculum Vitae}


%-------------------------------------------------------------------------------
\begin{document}

% Print the header with above personal information
% Give optional argument to change alignment(C: center, L: left, R: right)
\makecvheader[R]

% Print the footer with 3 arguments(<left>, <center>, <right>)
% Leave any of these blank if they are not needed
\makecvfooter
  {\today}
  {Ricardo Barrué ~~~·~~~ Research statement}
  {}

% Print the title with above letter information
\makelettertitle

%-------------------------------------------------------------------------------
%	LETTER CONTENT
%-------------------------------------------------------------------------------
\begin{cvletter}

The core of my research has been the study of the Higgs boson. By precisely measuring the properties of its interaction, we can search for small deviations not predicted by our current theory. I focused particularly on deviations to our current theory that may explain the matter-antimatter asymmetry in the Universe. My research has given me vast experience with advanced data analysis methods (inc. machine learning), which I describe in this statement, along with how this experience can be used to tackle common problems in epidemiological modelling. I will finalize with a forward-looking section, where I discuss several ways to apply my skillset for open research problems in epidemiological modelling.

%For this, I worked in several ATLAS analyses, which have given me quite some experience with advanced data analysis, including machine learning methods, which I describe in the section `Data analysis for Higgs boson measurements'. In order to circumvent some of the shortcomings of traditional analyses and improve the search of new physics, I explored several simulation-based/likelihood-free inference methods (based on neural networks), which I describe in the section `Simulation-based inference methods for searches of new physics in Higgs boson interactions'. Across both sections, I will detail how the developed tasks and acquired experience connect to challenges in epidemiological modelling. I will finalize with the `Future views' section, where I discuss several ways to apply my skillset for open research problems in epidemiological modelling.

% TODO: give more context from the physics goals and what has driven the research
% TODO: talk more about the impact of my work in the analyses/fields I worked on
% TODO (cover letter): add some pointers to the fact that we work with very large datasets from LHC experiments, so efficient software
% - from Claude: "Data handling: Emphasize your experience working with the large, complex datasets typical in high energy physics experiments like ATLAS. This is directly relevant to handling epidemiological data."
% 

\lettersection{Data analysis for Higgs boson measurements}

% My experience with data analysis started with an undergraduate summer internship where I studied the use of Boosted Decision Trees to separate signal events from Higgs boson pair production (very rare) from background events, which come from a myriad of similar, much more frequent processes. Such methods may be used for the task of detecting events from early signs of an outbreak above baseline health data. In this internship I also showed that retraining the BDTs after applying Principal Component Analysis (PCA), a dimensionality reduction technique, on the input variable set led an increase in separation performance (area under ROC curve $97.7\% \to 99.9\%$). Such a method can be used in early monitoring of outbreaks to determine the most relevant (linear) combination of variables, while maintaining explainability (given that it is merely a linear combination of the original input variables), relevant when this information is to be passed to decision-makers.

Production of high-ergy the region of phase space most sensitive to new physics contributions

During my masters, I contributed to a novel ATLAS search of very high-energy Higgs boson decays (rare process), which is  where I worked on defining the optimal signal identification and event selection criteria, to keep the ratio of signal to background events as high as possible. This again, relates to the task of detecting events from early signs of an outbreak. This was published in \href{https://doi.org/10.1016/j.physletb.2021.136204}{Physics Letters B 816 (2021) 136204}, and I was granted exceptional authorship for this publication (granted to incoming ATLAS members which made significant contributions to a publication).

During my PhD, I contributed to a precision measurement of Higgs boson decays. My main contribution was a derivation of (simulation-based) priors for several background uncertainties, finding regions of phase space where the impact of the same uncertainty in data and simulation was different (requiring additional degrees of freedom in the statistical model), and determining the contribution of these uncertainties to the uncertainty on final result. This task was a key ingredient to the final result of the analysis, published in \href{https://arxiv.org/abs/2410.19611}{arXiv:2410.19611} It can be directly translated to the challenge of uncertainty quantification in models of outbreak propagation using a mix of simulation (from e.g. mechanistic models or generative machine-learning models) and data.

In such an analysis, priors for modelling uncertainties are derived by comparing the nominal analysis sample (large number of events) with a sample from an alternative generator (low number of events). This analysis employs a method to create a proxy for the alternative sample from the nominal sample. The goal is to reduce the statistical fluctuations in the alternative sample, which could otherwise lead to overestimating the uncertainty. This method is a neural network-based high-dimensional reweighting technique, which I trained and evaluated for some of the dominant backgrounds in this analysis. Such a technique can be used in epidemiological modelling, to use well-understood distributions from previous outbreaks to inform the modelling in early stages of a new outbreak, where information is scarce.

I proposed and am now leading a novel ATLAS search based on the precision measurement described above. I contributed to the full analysis pipeline, from the choice of relevant observable (studied in previous theoretical literature) and the study of its modelling, to developing the statistical model. The skills acquired in this analysis translate almost directly into the task of outbreak modelling: selecting relevant observables (e.g. cross-border movements), studying their modelling and developing the statistical model in order to produce reliable results with proper uncertainty quantification.

\lettersection{Simulation-based inference methods for searches of new physics in Higgs boson interactions}

The analyses mentioned above use (frequentist) likelihood-based statistical inference and hypothesis testing procedures, but estimate the likelihoods from histograms of a small number of variables (since the full likelihood is intractable), which leads to suboptimal precision. During my PhD, I became curious about simulation-based (also called likelihood-free) inference methods, which aim at circumventing such a limitation. These use quality simulators of a process (which can be mechanistic or even a quality deep generative AI model) and neural networks to estimate the quantities relevant for statistical inference in in a high-dimensional, per-event basis.

In the context of the search for new physics in Higgs boson interactions mentioned in the previous section (and chronologically simultaneous to that research), I explored a method to estimate sufficient statistics of the likelihood. This work resulted in a publication - \href{http://dx.doi.org/10.1007/JHEP04(2024)014}{JHEP04(2024)014} (of which I was the main author) showing that such a method has a performance competitive with the traditional approach, while being robust to background processes (noise) and unobservable variables. I am now supervising a master student on the extension of that work, exploring a simulation-based inference method to estimate likelihood ratios using parametrized neural networks. % Mention that we do uncertainty quantification with ensembles and perform some validation via the position of the MSE for an Asimov dataset.

Such methods can be used in epidemiology when using a parametric description of outbreak propagation (usually based on mechanistic models). These allow proper parametric inference methods to be used, which also assists in uncertainty quantification, by plugging in the obtained values (and uncertainties) for the parameters into mechanistic models and run several simulations to obtain credible intervals for desired variables.

% \lettersection{Other work}

% Simultaneous to the research described above, I also worked on the real-time event selection (trigger) system of the ATLAS experiment, critical for analysis. I worked on the upgrade of this system for Run 3 (started mid-2022), starting by defining the optimal parameters for some of the reconstruction algorithms in the software-based trigger system using simulations, having further worked on the validation of my studies using 2022 data. This resulted in a publication \href{https://doi.org/10.1088/1748-0221/19/06/P06029}{JINST 19 P06029}.

\lettersection{Future views}

Particle physics has long had a story of using in its (data) analyses hybrid methods, which combine high-quality (mechanistic) simulators and advanced machine learning. This approach has also been explored in epidemiology, and I believe there is still large potential to be unlocked. 

One possible research direction in this vein is the exploration of simulation-based/likelihood-free inference methods. There is a growing body of work in deriving such methods based on a Bayesian formalism (vs. the frequentist one used in Particle Physics), simultaneous to large improvements on the quality of simulators of outbreak development (driven largely by the COVID-19 pandemic). Such developments, coupled to my knowledge experience with simulation-based inference methods, spark my desire to explore their application for outbreak modelling. In particular, because they allow for proper uncertainty quantification. % add reference

A topic that has recently grown in interest in particle physics research and that I think would be really interesting to further explore in the context of epidemiology is the design of differentiable mechanistic simulators, whose predicted output can be compared to the observed data and the simulator parameters updated using standard gradient descent techniques.  % add reference

Beyond methodology research, I also would like to focus on the operational use of such methods. For this, I believe that thinking about computational efficiency, robustness (to e.g. missing data), uncertainty quantification and explainability already in the design phase for such methods described above is crucial. % add reference

Additionally, I would like to engage my passion for communicating scientific and technological results to non-expert audiences to bridge the gap between the output of these methods and the decision-makers with quality communication based on clear language and informative visualizations.

\end{cvletter}


%-------------------------------------------------------------------------------
% Print the signature and enclosures with above letter information
\makeletterclosing

\end{document}