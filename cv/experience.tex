\cvsection{Research experience}

\begin{cventries}

    \cventry
    {LIP}
    {PhD researcher}
    {}
    {Sep. 2017 - PRESENT}
    {
        \textbf{Simulation-based inference methods for searches of new physics in the Higgs sector}\vspace{12pt}
        \begin{cvitems}      
            \item {spearheading the group's exploration of these methods, supervising a student on a study of likelihood ratio estimators, targetting BSM components in the HWW interaction in leptonic WH production.}
            \item {explored a method to estimate sufficient statistics of the likelihood, applied to the search for CP-violating BSM components in the HWW interaction in leptonic WH production (\href{https://doi.org/10.1007/JHEP04(2024)014}{J. High Energ. Phys. 2024, 14 (2024)} - main author), showing that such a method has competitive sensitivity to methods explored in previous literature.}
        \end{cvitems}
    }

    \cventry
    {ATLAS Collaboration}
    {Member}
    {}
    {Oct. 2018 - PRESENT}
    {
        \textbf{Higgs boson measurements}\vspace{14pt}
        \begin{cvitems}
            \item {leading the first ATLAS search for BSM CP-violating components in HWW vertex in WH production using angular observables, defining the best observable, performing reconstruction and categorization studies and developing the fit model.}
            \item {contributed to the combined measurements of Higgs boson decays to bottom and charm quarks in VH production, calculating priors for the shape modelling uncertainties of the W+jets background in the high-momentum regime as well as defining the correlation scheme for shape and acceptance uncertainties of the W+jets and Z+jets backgrounds.}
            \item {contributed to the first ATLAS search of Higgs boson decays to bottom quarks VH production in the high-momentum regime, defining the optimal Higgs boson identification strategy and proposing an event selection criteria that led to a 4\% improvement in analysis significance.}
        \end{cvitems}\vspace{19pt}
        %
        \textbf{Jet reconstruction at the ATLAS High-Level Trigger}\vspace{14pt}
        \begin{cvitems}
            \item {validated the performance of the jet trigger during the first data-taking year of LHC Run 3 (2022).}
            \item {contributed to the definition of the optimal track, vertex and Particle Flow reconstruction parameters for LHC Run 3.}
        \end{cvitems}
    }

    \cventry{LIP - ATLAS group}
    {Research Internship}
    {}
    {Sep. 2016 - Nov. 2016}
    {Developed Python and BASH scripts to parse results of tests of CPU/GPU implementations of jet clustering algorithms for the ATLAS trigger system, showing that a GPU implementation has superior performance to the CPU implementation and that a system with a mid-priced GPU (Quadro K2200) and a large amount of RAM (24GB) was the most cost-effective implementation.}

    \cventry{LIP - CMS group}
    {Research Internship}
    {}
    {Jul. 2016}
    {Explored the application of Boosted Decision Trees in the search for Higgs boson pair production, showing that performing Principal Component Analysis on a set of physics-motivated variables and using the transformed variable set as input variable led to a large improvement in performance (area under ROC curve $97.7\% \to 99.9\%$).}

    \cventry{Institute for Plasmas and Nuclear Fusion}
    {Research Internship - Control Systems Engineering}
    {}
    {Jul. 2015}
    {Implemented the control system for a remote-controlled radiation physics experiment using DSPIC microcontrollers.}

\end{cventries}