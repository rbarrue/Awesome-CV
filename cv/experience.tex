\cvsection{Research experience}

\begin{cventries}

    \cventry
    {LIP}
    {Researcher}
    {}
    {Sep. 2017 - PRESENT}
    {
        \textbf{Simulation-based inference methods for searches of new physics in Higgs boson interactions}\vspace{12pt}
        \begin{cvitems}      
            \item {spearheading the group's exploration of simulation-based-inference (likelihood-free inference) methods, proposing and supervising a study of a high-dimensional method to estimate the likelihood ratio, to be used directly in the statistical analysis.}
            \item {explored a high-dimensional method to estimate sufficient statistics of the likelihood, showing that it has competitive sensitivity to that from standard methods, based on a small set of physics-informed variables. Published on \href{https://doi.org/10.1007/JHEP04(2024)014}{J. High Energ. Phys. 2024, 14 (2024)} (main author).}
        \end{cvitems}
    }

    \cventry
    {ATLAS Collaboration}
    {Member}
    {}
    {Oct. 2018 - PRESENT}
    {
        \textbf{Data analysis in Higgs boson measurements}\vspace{14pt}
        \begin{cvitems}
            \item {leading a novel search for new physics in Higgs boson interactions using an angular observable. Contributed to the full analysis pipeline, from the definition of relevant observable and the study of its modelling, to developing the statistical model.}
            \item {contributed to a measurement of Higgs boson decays: used a high-dimensional reweighting method to reduce spurious (modelling) uncertainties due to statistical fluctuations in simulated samples with small number of events; contributed to the development of the statistical model. Pre-print in \href{https://arxiv.org/abs/2410.19611}{arXiv:2410.19611} (submitted to JHEP).}
            \item {contributed to a novel search of very high-energy Higgs boson decays (rare process): defined the optimal signal identification strategy and proposed an event selection criteria that led to a 4\% improvement in analysis significance. Published in \href{https://doi.org/10.1016/j.physletb.2021.136204}{Physics Letters B 816 (2021) 136204}.}
        \end{cvitems}\vspace{19pt}
        %
        \textbf{Real-time event selection with the ATLAS trigger system}\vspace{14pt}
        \begin{cvitems}
            \item {validated a subset of the algorithms of the software-based ATLAS trigger system with data from the first year of LHC Run 3 (2022). Published in \href{https://doi.org/10.1088/1748-0221/19/06/P06029}{JINST 19 P06029}.}
            \item {contributed to the definition of the optimal reconstruction parameters for a subset of the algorithms on the software-based ATLAS real-time event selection (trigger) system (which processes ~100k events/second), balancing computing time and selection efficiency, keeping the decision latency below 1s/event.}
        \end{cvitems}
    }

    \cventry{LIP - ATLAS group}
    {Research Internship}
    {}
    {Sep. 2016 - Nov. 2016}
    {Developed Python and BASH scripts for parsing results of tests of novel reconstruction algorithms for the ATLAS trigger system, helping to determine the most time- and cost-effective implementation.}

    \cventry{LIP - CMS group}
    {Research Internship}
    {}
    {Jul. 2016}
    {Explored the application of Boosted Decision Trees for signal-background separation in the search for Higgs boson pair production (very rare process). Showed that using Principal Component Analysis (dimensionality reduction method) on a set of physics-informed input variables and retraining the BDT using the transformed variable set as input led to an improvement in performance (area under ROC curve $97.7\% \to 99.9\%$).}

\end{cventries}