\cvsection{Publications}

% \cvparagraph{I mention my original contributions to each ATLAS publication and link the internal notes/supporting documents.}

\cvsubsection{Peer-reviewed}

\begin{cventries}

    \cventry
    {Measurements of $WH$ and $ZH$ Higgs production with decays into bottom quarks and direct constraints on the charm Yukawa coupling with 13 TeV collisions in the ATLAS detector}
    {The ATLAS Collaboration}{}
    {\href{https://doi.org/10.1007/JHEP04(2025)075}{J. High Energ. Phys. 2025, 75 (2025)}}
    {
        Internal note: \href{https://cds.cern.ch/record/2743096}{ATL-COM-PHYS-2020-765}
    }

    \cventry
    {The ATLAS Trigger System for LHC Run 3 and trigger performance in 2022}
    {The ATLAS Collaboration}
    {}
    {\href{https://doi.org/10.1088/1748-0221/19/06/P06029}{JINST 19 P06029}}
    {
        Internal note: \href{https://cds.cern.ch/record/2845056}{ATL-COM-DAQ-2022-133}
    }

    \cventry
    {Simulation-based inference in the search for CP violation in leptonic WH production}
    {\underline{Ricardo Barrué}, Valerio Dao, Patricia Conde Muíño, Rui Santos}
    {}
    {\href{https://doi.org/10.1007/JHEP04(2024)014}{J. High Energ. Phys. 2024, 14 (2024)}}
    {
        % \begin{cvitems}
        % \item {Explored a simulation-based inference method that uses deep neural networks and generator information to estimate sufficient statistics of the likelihood using detector-level observables (SALLY), showing that it has a sensitivity competitive with that of other observables defined previously in the literature.}
        % \end{cvitems}
    }

    \cventry
    {Measurement of the associated production of a Higgs boson decaying into $\mathbf{b}$-quarks with a vector boson at high transverse momentum in $\mathbf{pp}$ collisions at $\mathbf{\sqrt{s}= 13 \:\text{TeV}}$ with the ATLAS detector}
    {The ATLAS Collaboration}
    {}
    {\href{https://doi.org/10.1016/j.physletb.2021.136204}{Physics Letters B 816 (2021) 136204}}
    {
        Part of master thesis work - \textbf{granted exceptional ATLAS authorship}. Internal note: \href{https://cds.cern.ch/record/2688371}{ATL-COM-PHY-2019-1125}.
        % \begin{cvitems}
        %     \item {Contributions: Higgs boson tagging strategy and event selection optimization, production and validation of analysis tuples, analysis framework development and validation (internal note: \href{https://cds.cern.ch/record/2688371}{ATL-COM-PHY-2019-1125})}
        %     \item {Part of master thesis work - \textbf{granted exceptional ATLAS authorship}}
        % \end{cvitems}
    }
    
\end{cventries}

\cvsubsection{Pre-prints}

\begin{cventries}
    \cventry{Early Career Researcher Input to the European Strategy for Particle Physics Update: White Paper}
    {Jan-Hendrik Arling, Alexander Burgman, \textit{et al}}{}
    {\href{https://arxiv.org/pdf/2503.19862}{arXiv:2503.19862}}
    {
        Part of the working group on "Communicating the importance of particle physics", where I co-developed the survey that resulted in the statements and recommendations presented. Provided feedback throughout the editorial process. Publication endorsed by the Early Career Researcher Representatives to the European Commission for Future Accelerators (ECFA).
    }    
\end{cventries}

% \cvsubsection{ATLAS Conference notes}

% Reviewed and approved internally by the ATLAS collaboration, generally presented at conferences and usually superseded by an accompanying paper with additional results which is sent to a peer-reviewed journal for publication.

\cvsubsection{Proceedings}

\begin{cventries}
    \cventry{The ATLAS trigger system}
    {\underline{Ricardo Barrué}, on behalf of the ATLAS Collaboration}{}
    {\href{https://pos.sissa.it/450/253/}{PoS(LHCP2023)253}}
    {
        Proceedings of The Eleventh Annual Conference on Large Hadron Collider Physics (LHCP2023), May 2023.
    }
\end{cventries}

% \cvsubsection{In approval}

