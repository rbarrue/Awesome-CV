\cvsection{Outreach}

\begin{cventries}
    
    \cventry{``Contributing to ATLAS critical activities and what it means for your science and career"}{}
    {}{Mar. 2024}
    {\href{https://indico.cern.ch/event/1378416/overview}{Round table} with early career scientists sharing our experience as part of the (mission-critical) data-taking operations team of the ATLAS detector.}

    \cventry{``The Higgs Boson and our lives"}{}
    {}{Sep. 2023}
    {\href{https://www.cienciaviva.pt/divulgacao-cientifica/o-bosao-de-higgs-e-as-nossas-vidas}{Round table} with Fabiola Gianotti (director-general of CERN) and José Antão (Industrial Liason Officer Portugal-CERN) discussing the science developed at CERN (including the Higgs boson discovery), the associated technology and its impact on the life of the common citizen.}

    \cventry{Official CERN guide}{}
    {}{Feb. - Sep. 2023}
    {Guided visits to the Synchrocyclotron and ATLAS Visitor Center with up to 20 people simultaneously. I also had to accommodate different degrees of familiarity with CERN's activities within the same group.}

    \cventry{``The Universe in a formula"}{}
    {}{Jul. 2021}
    {Interview for Visão magazine (top 3 of magazine sales in Portugal) about theories of everything, alongside António Damásio (neuroscientist, author of the popular book \textit{Descartes' Error}) and other Portuguese scientists.}

    \cventry{``New subatomic particles and the fifth dimension"}{}
    {}{Feb. 2021}
    {\href{https://visao.pt/exameinformatica/videos-ei/eilive/2021-02-08-cromo-da-semana-novas-particulas-atomicas-e-a-quinta-dimensao/}{Live interview} (in Portuguese) for Exame Informática, the largest technology magazine in Portugal, describing to the general public the paper "A warped scalar portal to fermionic dark matter" (\href{https://doi.org/10.1140/epjc/s10052-021-08851-0}{Eur. Phys. J. C 81, 58 (2021)}). }

    \cventry{LIP brochure}{}
    {}{Jan. 2021}
    {Developed (along with LIP's outreach coordinator and designer) a brochure motivating the importance behind LIP's activities and some of the science behind them, to be presented at in-person outreach activities.}

    \cventry{90 Seconds of Science}{}
    {}{Sep. 2020}
    {Interview about my thesis work, condensed into a \href{https://www.rtp.pt/play/p2936/e491211/90-segundos-ciencia}{90-second clip played on national radio} (Antena 2).}

\end{cventries}